\chapter{Background \& Objectives}

This section should discuss your preparation for the project, including background reading, your analysis of the problem and the process or method you have followed to help structure your work.  It is likely that you will reuse part of your outline project specification, but at this point in the project you should have more to talk about. 

\textbf{Note}: 

\begin{itemize}
   \item All of the sections and text in this example are for illustration purposes. The main Chapters are a good starting point, but the content and actual sections that you include are likely to be different.
   
   \item Look at the document on the Structure of the Final Report for additional guidance. 
   
\end {itemize}

\section{Background}
What was your background preparation for the project? What similar systems or research techniques did you assess? What was your motivation and interest in this project? 
\subsection{Background and Preparation}
Background research was compromised of analysing existing systems like the original Access Aber, previous work completed by the developer that used map technology within Android and general research into the map technology to gather a strong understanding of exactly what was possible.

A majority of time was spent analysing Access Aber and the currently existing criticisms of it, a lot of the problems were clear without the user feedback that had been provided but some criticisms were more subtle but easy to understand like how the application provides no main menu as such causing the user to feel lost from the start. It was also obvious that at a technical level, the included features were not overly complex, at least not to complete on the Android platform. Due to the application originally being developed on a system which allowed it to be run both on iOS and Android there probably were limitations from both sides which leads to having to develop around the weak elements in both platforms. As stated developing for a single platform should help us avoid these problems while also giving us the ability to take advantage of what is there. Due to this a large amount of research was completed into both UI design within Android as well as the technical benefits both Open Street Maps and Google Maps could provide the development. 

Research was also completed into exactly how the mesh of locations would be created and represented, within the design specification it was concluded that a searchable graph was a possible solution and one worth researching further. This also meant analysing how other existing applications had mapped roads to graphs and the general area of map representation. While this gave a fair amount of answers further decisions needed to be made on both how to search the constructed graph and how to make it an extendible system for future additions. This was one of the main conditions brought up in meetings with the original 'customers' of the Access Aber application, it had to be functional after the initial developers had left. It was also described that it would be beneficial for the application to be developed in such a way that a possible future development that involved a central file store of the information for ranging versions of the application. Due to this further research was done on how we could leave the application in such a way that it facilitated the addition of this feature possibly without using it upon the current cycles completion. 

Past work by the developer was also analysed for anything of use that may come out of it, the application analysed was proven to have functional working code and as such was a place where possible solutions could be found. This included implementation of maps in an Android environment, the logging of a route which is something that has been previously outlined as a key requirement and a fair amount of small features relating the monitoring of a users location and the information which can be gathered from that. Most information gathered from this however was fairly irrelevant, it was decided it was not the best place for referencing in future. 

Finally the Android API and several Android libraries were examined for the benefits they could provide the development with, along with the search for several features that had already been selected for inclusion within the application. This ranged from simple research on Expandable List Views to research on the best way to present the application in an intuitive way, a lot of factors were gathered from the original feedback which guided a lot of the research. While a slide function for the screens was visually appealing the possibility of it confusing users was something that removed it as a possibility very quickly. Research was also performed into Google's Material Design program, a set of guidelines for good design within the Android environment. It contained a large amount of information relating to designing professional and good looking applications, while the information was too much to be fully applied in the projects time line some key ideas were taken from it relating to the design of elements within an application and general theme layout. Furthermore some sites were analysed for basic colour themes and how to implement a simplistic design without it looking unprofessional.

\subsection{Interest}
This project has been chosen due to a variety of reasons, revolving around both the developers interests and the possibility of the project leading to something that can have real world influences. The main source of interest is that the project allows for a development environment that takes input from the real world, mobile devices are of a great deal of interest to the developer due to the data that can be gathered from them and the manipulation of this data. Mobile devices are steadily changing the way we live, with applications that not only help us find where to go but tell us what we can do when we get there. Where other people have liked and even what is best suited to us based on past choices we have made and the device has noted. While some systems outside of mobile development clearly have massive real world influences it is felt by the developer that the easy entry and possibilities provided with mobile development make it one of the most accessible and revolutionary platforms.



\section{Analysis}
Taking into account the problem and what you learned from the background work, what was your analysis of the problem? How did your analysis help to decompose the problem into the main tasks that you would undertake? Were there alternative approaches? Why did you choose one approach compared to the alternatives? 

There should be a clear statement of the research questions, which you will evaluate at the end of the work. 

In most cases, the agreed objectives or requirements will be the result of a compromise between what would ideally have been produced and what was felt to be possible in the time available. A discussion of the process of arriving at the final list is usually appropriate.

\section{Research Method}
You need to describe briefly the life cycle model or research method that you used. You do not need to write about all of the different process models that you are aware of. Focus on the process model or research method that you have used. It is possible that you needed to adapt an existing method to suit your project; clearly identify what you used and how you adapted it for your needs.
