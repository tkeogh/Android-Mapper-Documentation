\chapter{Evaluation}


\section{Requirements and Process}
\subsection{Requirements}
The requirements, as stated before, were mostly correct. However it has been shown that one could have been a lot more clear, mainly down to a lack of research relating to that specific requirement. In future it would be wise to do more than skim the surface of what seems a simple requirement as it can be deceiving.  It could also be argued that there was a lack of detail surrounding some requirements, while it was fleshed out in design it does not give the designer much to build around. However there were requirements that had the sufficient detail, its just making sure that in future the developer makes all requirements equal in depth. While an area which may need improvement, requirements can be seen as a strong area in the report, with all key functions included.
\subsection{Process}
Working with the process has definitely been seen as a success in this project, with the process being adapted from other more concrete versions it was unsure if it would provide the developer with a strong enough base to develop from. However with design being completed twice it was more than sufficient, it also meant that once one feature was completed it gave a framework for further features. This was due to some problems already being encountered through other development cycles, a problem that occurs a second time is easy to fix. In hindsight the process may have been a little vague for a full team to follow, with a single developer it is easy to visualize something you've planned out in your head. The description included within the document could be more detailed, an area to focus on when working in a team. 
\section{Tools and Design}
\subsection{Tools}
With this project tool use was very inflexible, as seen in the document it was necessary to use a specific IDE to gather what were seen as needed skills if this was a field the developer wanted to enter in the future. GitHub was used for version control, again a standard tool that is likely to be used in almost any software project. One improvement which could have been made is very early planning on tools, by not doing this the developer took time away from learning Android Studio because they were not aware that it had become the preferred development environment compared to the now deprecated Eclipse with ADT. A solid effort was made to use the best tools, however it was not a complex task. 
\subsection{Design}
Design was another area which can be seen as a success in a way, but with definite room for improvement. The outline of the program is well set out, the initial class diagram looks like it could have been implemented to a working degree. While this was built upon in further iterations it worked as a general idea. With it being developed, the final design being in this document, it turned into a detailed description, including justification of what was built. While some of the application differs it was due to either further improvements on design or a change in plan. With design also including sequence diagrams, overall structure and flow diagrams a very strong outline of the project was built. In depth design was achieved through the pseudo code of important algorithms, some of which were literally translated to code line for line and worked with little to no issues.

An area where improvement could have been made was a better highlight of exactly $how$ the program would be extendible. While many references are made to the idea little detail is provided, causing effort to be put into design in the implementation phase which could have been done previously. 
\section{Implementation and Aims}
\subsection{Implementation}
Implementation was by far the strongest stage of development, it is thought this was down to solid requirements and design specifications. All the necessary features were set out with possible design solutions also there to rely on. Throughout it was also required to make the application easily expandable, an area that is felt to be the strongest of all features developed. While not a feature in itself, it should make the most different to future maintainers. Route Searching is also seen as a success, with grading and route plotting being relating features. With this current project the improvements on route finding has been huge compared to the original Access Aber\cite{aa}, which is what has been referenced for areas for improvement. 

However improvements can still be made, even with the areas considered strongest. A single graph for route finding would be a benefit, as well as better help function though more communication with the University would have to be undergone for some of the possible ideas previously outlined. Route plotting is another area with improvements in mind, to be outlined in the future work sections. 
\subsection{Compared to Aims}
When comparing the finished project to the initial aims, it is sufficient to say they have been mostly fully met, with improvements in some. While some are not implemented ideally, it was not expected in a project with a fairly short time span. With reference to the Requirements Specification, as has been mentioned previously, four out of the five outline requirements can be considered fully met with additions, with one being an outlier due to extenuating circumstances. 

\section{Future Work}
Future work is an area that could branch out into any number of directions, purely down to who is developing the project at the time. Currently the developer would like more time be put into the route finding implementation, with more support for disabled users and a more technically challenging solution to the graph situation. A set of possible future work, all of which interest the current developer are 
\begin{itemize}
   \item Implement a single graph and a more complex search algorithm.
   \item Allow users to input variables about themselves and find routes based on that.
   \item Separate the route plotter from the current application.
   \item Add a touch to log point feature, a user then has the choice to walk a route or not. 
   \item Further filters on buildings, department and theatres possible solutions. 
   \item Improve the Help function significantly, possibly implement a system which contacts helpers on open days. 
\end{itemize}

All of these are areas which will benefit the current application. While not all of them may get much use, they at least provide experience for the implementer and as such benefit someone. 

Providing features for disable users is something the developer feels strongly about, a separate list has been set out to demonstrate possible ideas. 

\begin{itemize}
   \item Multiple colour schemes for visually impaired users.
   \item Touch to talk text, for those who again have visual impairments.
   \item Read out instructions during movement on a route, for example 'turn left'. Sat nav like functionality.
   \item Log what a user says they are capable of and have them never have to specify again.  
   \item List buildings based on accessibility, find entry to buildings on routes based on accessibility. 
\end{itemize}

It is felt the application can be developed to be something truly useful for the University, if development were to continue the benefits it could provide to a range of users would make the development time worth it. 
\section{Self Evaluation}
Overall as a developer i am happy with the work i have completed and the standard it has been completed to. Working on a project this large is not something i have done before and as such it was an experience i was nervous about. One area i am particularly proud of is the adherence to the requirements set out, as well as meeting most of the smaller requirements mentioned in meetings. While progress has not always been smooth the amount of technical knowledge i have picked up along the way has made the development time worth it. While developing the speed at which features were being completed was much faster than anticipated at points, if a larger amount of research had gone into one or two of the features before hand they also would have had their development time cut down significantly. Once i was a good way into the project my familiarity and memory of the architecture of the application was a huge bonus, the only reason for having the in depth knowledge of how the application should work was due to the thorough documentation done before development. 

However there are areas i have noticed are lacking, one of these related to the reading of documentation. There were several times during development that the skimming of existing documentation for Android caused delays, one time for over two days. An area that definitely needs improving on. It has also been noted that as a developer i tend to be side tracked by features i did not set out to work on, doing this creates a sort of non organised progression and leads to design notes being ignored and an overall lack of cohesion until all of the changes that weren't meant to happen have been documented. However it is another area i should be able to easily change. A final area i feel i can improve on is the use of the tools available to me, while in this project it was a decision made to try and not rely on libraries at the end of the project i feel that may have been wrong. By refusing free help i think its possible i have detrimentally affected the project for my own gain when really i could have learnt what i wanted at a later date. 

In conclusion i am happy with what has been achieved, having found areas i can improve on is another bonus as it will help me become a better developer. I feel the major gain however is that i have now identified an area of development i really feel passionate about. While before the project i was a fan of mobile development and its potential power i am now planning future projects for the platform. Working for so long and barely scratching the surface of what's possible just through the Android API has made me want to keep developing and possible seek a career in Android development. 

