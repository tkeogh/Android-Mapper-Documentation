\thispagestyle{empty}

\begin{center}
    {\LARGE\bf Abstract}
\end{center}

The aim for this project was to provide an alternative android native application to the current
Access Aber web application, while also improving on the groundwork laid out by the initial
work. This was done due to the help it can provide to users who are visiting the University as
well as providing at least some assistance to those with disabilities. Aberystwyths campus can be
a hard place to navigate and the benefits provided from a mobile device can benefit users to a large
degree.
The application was developed for the android OS using the Android Studio IDE and relies
heavily upon the Google Maps API. It provides a range of services to the user, some of which
were brought over from the existing application and improved, some new. These include route
finding around campus, user refined display’s of building locations, route plotter with application
compatible outputs and a location based help system.
This dissertation covers the original analysis, requirements, design and implementation of the
project as well as details on the testing and design decision made throughout the development.
The conclusion is an android application which met its original requirements specification and
adds further complexity to the initial web version of the project. This includes route grading and
the representation of this along with an extendable solution to route finding using Google Maps
and graph searching techniques.